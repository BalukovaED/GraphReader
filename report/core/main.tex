\chapter{Проект}
\section{Средства реализации}
\noindent\indent Для реализации проекта были использваны следующие решения:
\begin{itemize}
  \item Язык программирования --- C++
  \item Парсер TinyXML для десериализации данных формата XML (OSM XML)
\end{itemize}

\section{Модули и алгоритмы}
\noindent\indent Алгоритм работы программы состоит в считывании OSM XML
файла и выводе информации в основной выходной поток в один проход путем
десериализации входных данных спомощью парсера TinyXML, небольшой предобработки,
заключающейся в рассчете расстояния между связанными узлами, и в потоковом выводе
освоенных данных, а также сохранении их во временный контейнер.
\par Для рассчета расстояния между узлами использовалась формула нахождения
расстояния между точками на сфере, где под сферой подразумевалась земная оболочка.
\subsection{Интерфейс}
\noindent\indent Интерфейс программного решения расположен в файле Graph.h и
представляет из себя класс \codepart{Graph}

\lstdefinestyle{customcpp}{
  belowcaptionskip=1\baselineskip,
  breaklines=true,
  frame=L,
  xleftmargin=\parindent,
  language=C++,
  showstringspaces=false,
  basicstyle=\footnotesize\ttfamily,
  keywordstyle=\bfseries\color{green!40!black},
  commentstyle=\itshape\color{purple!40!black},
  identifierstyle=\color{blue},
  stringstyle=\color{orange},
}

\lstdefinestyle{customasm}{
  belowcaptionskip=1\baselineskip,
  frame=L,
  xleftmargin=\parindent,
  language=[x86masm]Assembler,
  basicstyle=\footnotesize\ttfamily,
  commentstyle=\itshape\color{purple!40!black},
}

\lstset{escapechar=@,style=customcpp}

\begin{lstlisting}[caption=class Graph (Graph.h), style=customcpp]
...
class Graph {
public:
    Bound bound;

    Graph() = default;

    void ReadGraphFromXML(std::string input);

    void clear();
private:
    void ReadXmlBounds(TiXmlElement* element);

    void ReadXmlNode(TiXmlElement* element);

    void ReadXmlTags(TiXmlElement* element, Node &n);

    void ReadXmlWay(TiXmlElement* element);
public:
    std::map<std::string, Node> nodes;
};
...
\end{lstlisting}
\par Основным методом данного класса является \codepart{ReadGraphFromXML},
принимающий на вход путь до OSM XML файла.
\par При необходимости можно считать данные из нескольких файлов в один граф,
путем вызова данного метода на каждом из файлов, содержащих необходимую для
считывания информацию.
\par Для <<очистки>> графа есть метод \codepart{clear}.

\subsection{Реализация}
\noindent\indent Первоначально исходный файл "передается" (файл открывается уже
внутри TinyXML - передается только путь к файлу) в качестве параметра парсеру, и
в дальнейшем файл поэлементно десериализуется.

\begin{lstlisting}[caption=method ReadGraphFromXML (Graph.cpp), style=customcpp]
...
void Graph::ReadGraphFromXML(std::string input) {
    //Setup TinyXML parser
    TiXmlDocument doc(input.c_str());

    if (!doc.LoadFile()) {
        std::cout << "ERROR: Could not load XML file: " + input << std::endl;
        return;
    }

    TiXmlHandle hDoc(&doc);

    //Start "recording" data
    TiXmlElement *pRoot = doc.FirstChildElement(OSM);

    if (pRoot == nullptr) return;

    TiXmlElement *pElem = pRoot->FirstChildElement();

    if (pElem == nullptr) return;

    //Read info about bounds (if it exists)
    if (!strcmp(pElem->Value(), BOUNDS)) {
        ReadXmlBounds(pElem);
        pElem = pElem->NextSiblingElement();
        if (pElem == nullptr) return;
    }

    //Read info about ways & nodes
    do {
        if (!strcmp(pElem->Value(), NODE)) {
            ReadXmlNode(pElem);
            continue;
        }
        if (!strcmp(pElem->Value(), WAY))
            ReadXmlWay(pElem);
    } while ((pElem = pElem->NextSiblingElement()) != nullptr);
}
...
\end{lstlisting}
\par Для обработки данных о каждом узле используется метод
\codepart{ReadXmlNode}, внутри которого считывается и выводится в стандартный
выходной поток его содержимое.
\par Для упрощения записи, в листинге 2.3 показано сокращение для повторяющихся
блоков записи и вывода параметров считываемого узла (вместо \codepart{\%TOKEN\%}
подставляются все возможные тэги, используемые для описания узла)
\begin{lstlisting}[caption=method ReadXmlNode (Graph.cpp), style=customcpp]
...
void Graph::ReadXmlNode(TiXmlElement* element) {
    if (element != nullptr) {
        Node n;
        n.id = element->Attribute(PARAM_ID) == nullptr ? "" : element->Attribute(PARAM_ID);
        std::cout << "NODE ID: " << n.id << std::endl;
        const char* buff;
        ...
        buff = element->Attribute(PARAM_%TOKEN%);
        if (buff != nullptr) {
            n.params[PARAM_%TOKEN%] = buff;
            std::cout << "\t%TOKEN%: " << buff << std::endl;
        } else n.params[PARAM_%TOKEN%] = "";
        ...

        TiXmlElement *pTag = element->FirstChildElement(TAG);

        //Read all tags from node's "children"
        if (pTag != nullptr) ReadXmlTags(pTag, n);

        nodes[n.id] = n;
    }
}
...
\end{lstlisting}
\par Для обработки данных о путях используется метод \codepart{ReadXmlWay}, суть
которого заключается в последовательном считывании информации о том, через какие
узлы данный путь проходит, а также в рассчете расстояния между соседними узлами.
\begin{lstlisting}[caption=method ReadXmlNode (Graph.cpp), style=customcpp]
...
void Graph::ReadXmlWay(TiXmlElement* element) {
    const char* id = element->Attribute(PARAM_ID);
    std::cout << "WAY: " << (id == nullptr ? "" : id) << std::endl;

    int i = 0;
    std::string first;
    element = element->FirstChildElement();

    do {
        //Read WAY TAG
        if (!strcmp(element->Value(), TAG)) {
            const char* key = element->Attribute(TAG_KEY);
            const char* val = element->Attribute(TAG_VALUE);
            std::cout << "\tTAG: "
                      << (key == nullptr ? "": key)
                      << " > "
                      << (val == nullptr ? "": val) << std::endl;
            continue;
        }
        //Read NODE id & calculate distance
        if (strcmp(element->Value(), ND)) break;

        if (i == 0) {
            first = element->Attribute(ND_REF) == nullptr ? "" : element->Attribute(ND_REF);
            std::cout << "\tNODE: " << first << std::endl;
            ++i;
            continue;
        }
        double cost = 0;
        /** Average distance
         * L = d*R
         * R = 6371km
         * cos(d) = sin(p_a)*sin(p_b) + cos(p_a)*cos(p_b)*cos(l_a-l_b)
         * l - lon, p - lat
         */
        const char* ref = element->Attribute(ND_REF) == nullptr ? "" : element->Attribute(ND_REF);
        double p_a = std::stod(nodes[first].params[PARAM_LAT]);
        double p_b = std::stod(nodes[element->Attribute(ND_REF)].params[PARAM_LAT]);
        double l_a = std::stod(nodes[first].params[PARAM_LON]);
        double l_b = std::stod(nodes[element->Attribute(ND_REF)].params[PARAM_LON]);
        double d = sin(p_a) * sin(p_b) +
                   cos(p_a) * cos(p_b) *
                   cos(l_a - l_b);
        d = acos(d);
        cost = d * 6371;//km

        nodes[first].neighbors.emplace_back(std::make_pair(ref, cost));
        nodes[ref].neighbors.emplace_back(std::make_pair(first, cost));

        std::cout << "\tNODE: " << ref << std::endl
                  << "\t\tDIST: " << cost << std::endl;

        first = ref;
        ++i;
    } while((element = element->NextSiblingElement()) != nullptr);
}
...
\end{lstlisting}

\chapter{Тестирование}
% \noindent\indent Для тестирования разработанного приложения был написан генератор тестовых изображений, представляющих из себя прямоугольные черные области с нанесенными на них белыми однородными объектами одного радиуса.
% \section{Синтетические тесты}
% \noindent\indent По результатам данного тестирования выявлено, что экстремумы согласуются с характерными размерами объектов на изображении. В частности, первый экстремум находится на позиции размера наиболее часто встречаемыми областями изображения(или на позиции среднего между самыми встречаемыми областями).
%
% \noindent\indent Для подтверждения данного результата привожу примеры тестовых изображений и графики функций вариограмм, построенных на их основе.
%
% \begin{figure}[!h]
%   \centering
%     \includegraphics[scale=1]{img_10_5.png}
%   \caption{Тестовое изображение с диаметром областей равным 20 и расстоянием между областями равным 5}
%   \label{fig:img_10_5}
% \end{figure}
%
% \begin{figure}[!h]
%   \centering
%     \includegraphics[scale=0.5]{graph_10_5.png}
%   \caption{График функции вариограммы, построенный на основе рис.  \ref{fig:img_10_5}}
%   \label{fig:graph_10_5}
% \end{figure}
%
%
% \begin{figure}
%   \centering
%     \includegraphics[scale=1]{img_10_10.png}
%   \caption{Тестовое изображение с диаметром областей равным 20 и расстоянием между областями равным 10}
%   \label{fig:img_10_10}
% \end{figure}
%
% \begin{figure}
%   \centering
%     \includegraphics[scale=0.5]{graph_10_5.png}
%   \caption{График функции вариограммы, построенный на основе рис.  \ref{fig:img_10_10}}
%   \label{fig:graph_10_10}
% \end{figure}
%
% \begin{figure}
%   \centering
%     \includegraphics[scale=1]{img_10_15.png}
%   \caption{Тестовое изображение с диаметром областей равным 20 и расстоянием между областями равным 15}
%   \label{fig:img_10_15}
% \end{figure}
%
% \begin{figure}
%   \centering
%     \includegraphics[scale=0.5]{graph_10_15.png}
%   \caption{График функции вариограммы, построенный на основе рис.  \ref{fig:img_10_15}}
%   \label{fig:graph_10_15}
% \end{figure}
%
% \begin{figure}
%   \centering
%     \includegraphics[scale=1]{img_10_20.png}
%   \caption{Тестовое изображение с диаметром областей равным 20 и расстоянием между областями равным 20}
%   \label{fig:img_10_20}
% \end{figure}
%
% \begin{figure}
%   \centering
%     \includegraphics[scale=0.5]{graph_10_20.png}
%   \caption{График функции вариограммы, построенный на основе рис.  \ref{fig:img_10_20}}
%   \label{fig:graph_10_20}
% \end{figure}
%
% \begin{figure}
%   \centering
%     \includegraphics[scale=1]{img_10_25.png}
%   \caption{Тестовое изображение с диаметром областей равным 20 и расстоянием между областями равным 25}
%   \label{fig:img_10_25}
% \end{figure}
%
% \begin{figure}
%   \centering
%     \includegraphics[scale=0.5]{graph_10_25.png}
%   \caption{График функции вариограммы, построенный на основе рис.  \ref{fig:img_10_25}}
%   \label{fig:graph_10_25}
% \end{figure}
%
% \begin{figure}
%   \centering
%     \includegraphics[scale=1]{img_10_20_err1.png}
%   \caption{Тестовое изображение с диаметром областей равным 20 и расстоянием между областями не более 15}
%   \label{fig:img_10_15}
% \end{figure}
%
% \begin{figure}
%   \centering
%     \includegraphics[scale=0.5]{graph_10_20_err1.png}
%   \caption{График функции вариограммы, построенный на основе рис.  \ref{fig:img_10_15}}
%   \label{fig:graph_10_15}
% \end{figure}
%
% \begin{figure}
%   \centering
%     \includegraphics[scale=1]{img_10_20_err2.png}
%   \caption{Тестовое изображение с диаметром областей равным 20 и расстоянием между областями не более 15}
%   \label{fig:img_10_15}
% \end{figure}
%
% \begin{figure}
%   \centering
%     \includegraphics[scale=0.5]{graph_10_20_err2.png}
%   \caption{График функции вариограммы, построенный на основе рис.  \ref{fig:img_10_15}}
%   \label{fig:graph_10_15}
% \end{figure}
%
% \section{Реальные тесты}
% \noindent\indent В качестве реальных данных для тестирования приложения использовались спутниковые изображения, содержащие в себе области с сильной кучностью облаков.
%
% \noindent\indent Результаты данного вида тестирования подтверждают правильность работы приложения.
